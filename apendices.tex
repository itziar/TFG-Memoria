\cleardoublepage
\appendix
\chapter{Instalación y Uso}
\label{app:instalacion}


Para el funcionamiento de la aplicación lo primero que hay que hacer es instalar todos los paquetes necesarios. Estos paquetes se encuentran en el archivo ''requirements.txt''. Por lo tanto con ejecutar en una shell ''pip install –r requirements.txt'' se instalaran los paquetes necesarios, sino es que ya están instalados en la máquina.


Una vez instalados, ejecutar en una Shell el comando ''python manage.py createdb'', que creará una base de datos, creando además un superusuario-administrador. A continuación hay que ejecutar el comando ''python manage.py runserver''. Cuando la aplicación esté en funcionamiento nos dirá tanto la IP como el PUERTO, por defecto localhost 8000. Se abrirá un navegador web donde habrá que introducir la URL ''htttp://localhost:8000''

\cleardoublepage
\chapter{Publicación}
\label{app:publicacion}

La aplicación se encuentra publicada en \url{http://54.71.52.158:8080/}