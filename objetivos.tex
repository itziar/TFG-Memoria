\cleardoublepage
\chapter{Objetivos}
\label{chap:objetivos}

En esta sección se pretende recoger todos los objetivos que han motivado la realización y desarrollo de esta herramienta.


\section{Objetivo general}
\label{sec:objetivo-general}
El objetivo de este trabajo se basa en el desarrollo de una aplicación web para la creación y gestión de toda clase de eventos. 

\section{Objetivos específicos}
\label{sec:objetivos-especificos}

Los principales objetivos que han guiado el desarrollo han sido:
\begin{itemize}
\item Aplicación universal: este es uno de los objetivos más importantes;  desarrollar una aplicación que pueda utilizarse desde cualquier plataforma, para que todos los  usuarios puedan acceder desde sus dispositivos.
\item Sencillez en su uso: la aplicación será utilizada por una gran variedad de personas, algunas de las cuales pueden no estar muy familiarizadas con las nuevas tecnologías, por lo que debe ser fácil de utilizar. Tiene que ser un diseño sencillo e intuitivo que sea atractivo y que incite el uso de la aplicación web.
\item Gestionar y promocionar mejor las actividades para que los alumnos crezcan tanto personal como profesionalmente. También se conseguirá que los alumnos tengan algo más de voz sobre las actividades que desean que se realicen, esto se consigue mediante el apartado de propuestas.
\item Web colaborativa: toda la información contenida en esta aplicación web deberá ser proporcionada por los usuarios de la misma tanto profesores creando los distintos eventos como los alumnos creando sus propuestas.
\item Aportar un elemento social: debido al gran interés que despiertan las redes sociales, es una buena idea integrar alguna función social para atraer a los usuarios, es importante agradarles para conseguir un mejor despliegue; muchos proyectos acaban fracasando por no tener un buen apoyo social inicial, produciéndose un reclamo por parte de los usuarios para volver al sistema tradicional.
\item Uso de tecnologías web avanzadas: en la actualidad el desarrollo web está creciendo a pasos agigantados. En este trabajo se pretenden utilizar tecnologías web muy utilizadas como lo son HTML5, CSS3 y Bootstrap y aprovechar sus funcionalidades.
\end{itemize}