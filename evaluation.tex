\cleardoublepage
\chapter{Evaluación del resultado}
\label{chap:evaluation}


El presente proyecto pretendía ofrecer una nueva herramienta para la gestión y promoción de los distintos eventos fuera y dentro de la Universidad objetivo que se ha conseguido.


Analizando cada uno de los objetivos descritos en el capítulo 2 también obtenemos un resultado satisfactorio:


\begin{itemize}
\item Aplicación universal: se ha conseguido crear una aplicación compatible con los distintos dispositivos, sistemas operativos y navegadores web existentes. Además, también se adapta para ofrecer un control y apariencia adecuados independientemente del tipo de dispositivo utilizado.
\item Sencillez en su uso: se ha creado un interfaz sencilla e intuitiva para todos los usuarios. Además presenta una página de contacto para poder comunicarse con el/los administrador/es.
\item Gestionar y promocionar mejor las actividades para que los alumnos crezcan tanto a nivel personal como profesional.  Además de las distintas actividades en cada una de las secciones, se ha proporcionado una herramienta para que los alumnos puedan proponer sus propias actividades (dentro de la sección propuesta) y los profesores si lo ven viable, esas propuestas se conviertan en futuras actividades. Con esta herramienta conseguimos que los profesores puedan escuchar lo que los alumnos más demandan.
\item Web colaborativa: aunque de momento no hay muchos usuarios (los creados por mí), esta herramienta puede ser un complemento al actual Moodle o incluso en un futuro llegar a sustituirlo añadiendo más módulos. Actualmente esta aplicación cuenta con la gestión de todos los eventos que se quieren hacer llegar a los alumnos y además consiguiendo algo que muy pocas veces ocurre y es que los alumnos puedan proponer sus ideas y salgan adelante en forma de seminarios, cursos, etc.
\item Aportar un elemento social: se ha creado un foro para que todos los miembros de la universidad puedan comunicarse y colaborar para lograr un mejor entorno universitario. 
\item Uso de tecnologías web avanzadas: se han utilizado varias herramientas web como HTML5, CSS3, Python, Django, JS, etc. que juntándolas han propiciado un buen resultado final.
\end{itemize}

