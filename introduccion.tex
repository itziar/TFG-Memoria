\chapter{Introducción}
%\addcontentsline{toc}{chapter}{Summary} % si queremos que aparezca en el índice
\label{sec:intro}

En este primer capítulo incluye una sección de presentación del presente trabajo; también se aborda un pequeño epígrafe sobre las motivaciones que han propiciado la elaboración del mismo, así como se concluirá con una reseña a la estructura que sigue este proyecto.


\section{Presentación}

Este documento describe el proyecto realizado para el trabajo de fin de grado en ingeniería en tecnologías de la telecomunicación. El proyecto consiste en la creación de una aplicación web para la creación y la gestión de eventos. Este proyecto pretende dar un servicio actualmente no existente o si existiera algo, no usable en el momento de elaboración de este TFG.


Por lo tanto, con esta nueva aplicación conseguiremos mejorar la gestión y promocionar los distintos eventos relacionados con la universidad como por ejemplo ofertas de trabajo y prácticas en empresas, cursos, becas, concursos, seminarios, etc.


El proyecto ofrecerá un servicio deslocalizado con el que se podrá acceder desde cualquier sitio y en cualquier  momento. 


En cuanto a los usuarios, la aplicación distingue varios tipos. Por un lado, están los usuarios no registrados que únicamente pueden ver los distintos eventos. Por otro, están los usuarios registrados que pueden acceder a una mayor funcionalidad dentro de la aplicación. Entre estos últimos se encuentra el usuario alumno, el usuario profesor y el usuario administrador. Los usuarios alumno, profesor y administrador pueden realizar acciones tales como añadir y comentar un evento además de participar en el foro. Además los usuarios del grupo de profesor y administrador podrán crear los distintos eventos y asignar los eventos denominados como actividades a otros usuarios.


Para crear la aplicación se ha utilizado el gestor de contenidos Mezzanine y como base de datos SQLite3 para el entorno de desarrollo en local y MySQL para el entorno de producción en Amazon Web Service. Se ha escogido este tipo de herramientas por su fácil acceso y disponibilidad así como por su sencillez tanto de instalación como de uso para usuarios no técnicos.
Para su despliegue se ha utilizado Amazon Web Services (aws). Se ha escogido esta herramienta por ser muy usada y ofrecer sus servicios gratuitos durante 1 año. Se ha escogido Ubuntu 14.04 LTS como sistema operativo del servidor.


Para finalizar, quiero señalar un par de puntos. Para mí, ya que el TFG supone el último paso para obtener un premio tan merecido como ansiado, creo que cada TFG debe llevar una parte de cada uno. Por este motivo, tanto la elección de colores como el logotipo y el nombre de la presente aplicación tienen un significado especial. 


Los colores elegidos tanto el azul como ese rosa en particular son colores que me atraen bastante. Y dado que me gustan no podía no incluirlos como colores principales.
Respecto al nombre de la aplicación, debo ser sincera y decir que me costó encontrar un nombre que me gustara y que pudiera tener un significado especial para mí. Había muchos pretendientes pero ninguno me llamaba la atención, hasta que encontré esa palabra alemana que además de sonar bien significaba algo para mí. Lo he llamado \textit{Welpe}, que para el lector que no esté familiarizado con el alemán viene a significar cachorro de perro, con lo que era todo un homenaje a mi perra. El logotipo sería mucho más sencillo con un nombre elegido. Si el proyecto se llamaba Welpe, la mejor idea sería una huella de perro.



\section{Motivación Personal}
La motivación personal de este proyecto residen en que ya que no existe ninguna herramienta usada por la universidad, todos los eventos llegan mediante un simple anuncio en el foro de la universidad y para el registro en alguna de las actividades, ésta se realiza mediante el envío de un correo electrónico o rellenando un simple ''google forms'', además si estamos realizando una actividad para que los asistentes puedan tener acceso al material de estudio es prácticamente imposible sin copiar correo a correo o escribir un anuncio en el foro donde los demás usuarios no registrados en esa actividad lo verían. 


\section{Estructura de la memoria}
Esta sección trata de detallar la estructura que se sigue a lo largo del proyecto para facilitar su orden y mejor comprensión.


\begin{itemize}
\item Capítulo I. Introducción: se realiza una descripción general del proyecto mostrando la motivación para el desarrollo del mismo y finalizando con una breve explicación de la estructura del mismo.
\item Capítulo II. Objetivos: se muestran cuál han sido los objetivos generales y específicos que se pretende lograr con la ejecución de este trabajo.
\item Capítulo III. Estado del arte: describe las distintas tecnologías utilizadas en la realización de la aplicación web.
\item Capítulo IV. Diseño e implementación: se describe la aplicación diseñada además se describe su estructura y su funcionamiento.
\item Capítulo V. Resultados: se analizan los resultados obtenidos.
\item Capítulo VI. Conclusiones:  se comentan las conclusiones finales tras la realización del trabajo y se comentan posibles líneas de desarrollo futuras.
\end{itemize}

