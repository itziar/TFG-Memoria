\cleardoublepage
\chapter{Conclusiones}
\label{chap:conclusiones}


\section{Conocimientos aplicados} 
\label{sec:aplicados}


Para el desarrollo del presente proyecto he utilizado los conocimientos adquiridos durante mi etapa en la Universidad cursando el Grado en Ingeniería en Tecnologías de la Telecomunicación.


Durante mi etapa en la universidad he crecido como profesional pero también como persona. Puedo decir que he adquirido gran cantidad de conocimiento, con lo cual quedarme con una sola asignatura para reflejar este proyecto sería un poco injusto; ya que de cada asignatura me llevo aunque sólo sea una pequeña parte y esa parte al final queda reflejada en lo que se hace día a día.


Si tuviera que destacar alguna o algunas asignaturas serían las orientadas en el área de programación puesto que su contenido y conocimiento se ajustan más al contexto del presente trabajo:



\begin{itemize}
\item Fundamentos de la Programación (FDP): es la asignatura que proporciona la base de la programación. Es donde ves por primera vez cómo se usa “eso de los ordenadores” y donde te dan ganas de salir corriendo (pero no lo haces).
\item Servicios y Aplicaciones Telemáticas (SAT): es la asignatura en que por primera vez ves que es útil eso de la programación y se pueden hacer cosas “chulas” e interesantes ya que en las anteriores asignaturas del área de programación eran mediante una terminal. En esta asignaturas aprendí Python, Django y también a crear webs. Con esta asignatura también descubrí que realmente sí que me gustaba programar (antes de esta asignatura, para mí, las asignaturas de esta área eran una continua tortura)
\item Ingeniería de Sistemas de Información (ISI): es la asignatura donde aprendes a desarrollar junto a tu equipo con herramientas que en el mundo profesional real se usan. A ver que hay vida más allá de las BBDD relacionales, etc. Aunque estas herramientas no las he usado en mi proyecto, las tengo muy presente en mi carrera profesional.
\item Desarrollo de Aplicaciones Telemáticas (DAT): es una asignatura donde aprendes a comunicarte (y sobre todo pegarte) con distintas APIs. 
\item Del resto de asignaturas me quedo con la capacidad de esfuerzo y entrega que he tenido, intentando dar al máximo en cada una de ellas, algunas veces viendo recompensa otras veces había que seguir intentándolo con mucho más empeño y esfuerzo.
\end{itemize}


También he aprendido bastante de los seminarios impartidos por profesores e incluso me ha tocado a mí dar alguno. En el que impartí sobre LaTeX, aprendí un poco más sobre esa herramienta que para muchos es desconocida además aprendí el trabajo y esfuerzo que hay al organizar un seminario que van más allá de una simple presentación.


Aunque no se promocione mucho, también he aprendido bastante en los diferentes cursos llamados MOOCs (Massive Online Open Courses) que ofrecen las diferentes universidades del mundo y plataformas. Empezó siendo un entretenimiento y al final se convirtió en un apoyo para mis estudios (aunque hubiera cursado alguno que no fuera directamente de la rama ‘Computer Science’). Con estos cursos además del propio contenido y de inglés, he aprendido a abrir mi mente y no estar encerrada en mi burbuja.


Quiero terminar este apartado con una GRACIAS a todos y cada uno de los profesores. De cada uno de vosotros me llevo una parte, tanto personal como profesionalmente. Seguro que aplicaré todos estos conocimientos proporcionados a lo largo de mi carrera profesional.



\section{Conocimientos adquiridos} 
\label{sec:adquiridos}

Durante la realización de este proyecto, no sólo he aprendido a poner en práctica mis conocimientos adquiridos durante la carrera, sino que este trabajo me ha permitido adquirir nuevos conocimientos y experiencias.


Al iniciar este trabajo, no me había enfrentado anteriormente a la creación de una aplicación de tal envergadura. Sin embargo, me he demostrado que mis conocimientos adquiridos durante la carrera (esfuerzo y constancia) estaban ahí que por mucho que diga “no puedo más” o “no lo voy a conseguir” siempre voy un poco más allá y siempre doy ese último pequeño esfuerzo.


En cuanto a mis conocimientos, he solidificado y madurado mis conocimientos en todas las tecnologías utilizadas y descritas en el capítulo 3.


También he aprendido sobre el mundo de los gestores de contenido. Lo útiles que pueden llegar a ser y lo rápido que crecen hoy en día.


Para el desarrollo he utilizado el IDE PyCharm y he decir que es una gran ayuda a la hora de depurar código y sobre todo a la hora de formatearlo (seguro que he infrautilizado este IDE pero en un futuro pienso utilizarlo mucho más).


Al decidirse desplegarlo en la nube, he adquirido conocimientos en \textit{cloud} que seguro serán muy útiles en mi carrera profesional, además también me incita a aprender un más sobre este tema tan de moda y tan olvidado en la universidad. 



\section{Trabajos futuros} 
\label{sec:futuro}


\begin{itemize}
\item Autenticación mediante LDAP de la Universidad
\item Ampliación de esta aplicación con nuevas sub-aplicaciones para poder ser más completo y que en un futuro pueda sustituir a Moodle o al menos que puede equipararse al nivel de Moodle.
\item Mejor eficiencia: siempre puede mejorarse la eficiencia de la aplicación, y debería mejorarse sobre todo las peticiones a la base de datos para así tener una mayor velocidad de navegación. 
\item Aplicación móvil: una aplicación móvil que ofrezca todas las funcionalidades que ofrece la aplicación web. Los dispositivos móviles son muy usados por cualquier persona joven y podría llegar a más alumnos. Para los más tradicionales, en cuanto a tecnologías se refiere, seguiría teniendo la aplicación web.
\end{itemize}
